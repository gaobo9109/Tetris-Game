\documentclass{article}

\addtolength{\oddsidemargin}{-.875in}
\addtolength{\evensidemargin}{-.875in}
\addtolength{\textwidth}{1.75in}

\addtolength{\topmargin}{-.375in}
\addtolength{\textheight}{1.75in}

\usepackage{titling}

% \pretitle{\begin{center}\Huge\bfseries}
% \posttitle{\par\end{center}\vskip 0.5em}
% \preauthor{\begin{center}\Large\ttfamily}
% \postauthor{\end{center}}
% \predate{\par\large\centering}
% \postdate{\par}

\title{CS3243 Tutorial 1}
\author{Varun Gupta(INSERT OTHER NAMES)\\A0147924X}
\date{26-01-2017}

\begin{document}

	\maketitle
	\thispagestyle{empty}
	\vspace{5mm}

    \section{Introduction}
    Tetris is likely to be one of the world's most famous and popular games.
    In this report, we describe how we devise an agent to play the game of Tetris.
    We use an agent that greedily picks the best possible next state from a given state,
    whilst using a heuristic function to approximate the value of a state. To train our heuristic
    function, we use a novel algorithm that is a combination of the well known Genetic Algorithm
    and Particle Swarm Algorithm. Our agent manages to clear 5 million lines on average with a max
    of 12 million lines, demonstrating that our algorithm is effective.

    \section{Agent Strategy}
    Our agent uses a linear weighted sum of features as the heuristic function for a given state. Given a state and a piece, the agent computes
    the heuristic function for all the possible next states, and then greedily picks the next state with
    the maximum heuristic value.

    //Insert Math equation

    \section{Features}
    We used the following features for our heuristic function:
    \begin{itemize}
        \item Altitude Difference: The difference between the height of the highest
        column and the height of the lowest column
        \item Number of Columns With Holes: The number of columns with holes, where a hole
        is defined as an empty square directly beneath a filled square
        \item Height of the highest column
        \item Number of holes in the entire board
        \item Number of wells: The number of columns that have a height less than that of the 2
        adjacent columns
        \item Rows cleared: The number of rows cleared for that particular move
        \item Total Column Height: The sum of the heights of all the columns
        \item Total Column Height Difference: The sum of the difference of heights between adjacent columns
        \item Column Transition:
        \item Deepest Well:
        \item Row Transition:
        \item Weighted Block:
        \item Sum of all Wells:
    \end{itemize}

    While running our training algorithms, we noticed that some features were more important an others. (To be continued)

    \section{Our Algorithm}

    (Someone please fill this)

    \section{Experiments and Analysis}

    Diagram 1: Learning
    Diagram 2: Performance of agent

    Experimentation
        1. Architecture specifications on which the algorithm was implemented
        2. The time taken to train
        3. The performance of the agent

    Analysis

    \section{Scaling to Big Data}

    Talk about parallelising algorithm. Mention MPI
    Get speedup

    \section{Conclusion}

    \section{References}



\end{document}
